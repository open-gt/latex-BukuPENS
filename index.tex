%
% Template Laporan Politeknik Elektronika Negeri Surabaya
%
% @author  Atqa Munzir
% @version 1.0.0
% @edit by
%  -
%




%-----------------------------------------------------------------------------%
% Configuration
%-----------------------------------------------------------------------------%

% Tipe dokumen adalah report dengan satu kolom.
%
\documentclass[10pt, a5paper, onecolumn, twoside, final]{report}


% Mengatur agar bagian bawah halaman tidak rata
%
\raggedbottom


% Load konfigurasi LaTeX
%
\usepackage{_internals/pens}


% Load settings untuk laporan yang sedang dibuat
%
%-----------------------------------------------------------------------------%
% Judul Dokumen
%-----------------------------------------------------------------------------%

% Judul laporan.
%
\def\judul{Open Source On-Premises Analytics Game Service}




%-----------------------------------------------------------------------------%
% Tipe Dokumen
%-----------------------------------------------------------------------------%

% Tipe laporan, dapat berisi: Kerja Praktik, Proyek Akhir, atau Tesis
%
\def\type{Proyek Akhir}




%-----------------------------------------------------------------------------%
% Informasi Penulis
%-----------------------------------------------------------------------------%

% Tuliskan nama Anda
%
\def\penulis{Atqa Munzir}


% Tuliskan NRP Anda
%
\def\nrp{5221600034}


% Tuliskan jenjang studi penulis
% dapat berisi: Diploma, Sarjana Terapan, atau Magister Terapan
%
\def\jenjang{Sarjana Terapan}


% Tuliskan program studi penulis
%
\def\prodi{Teknologi Game}


% Tuliskan jurusan penulis
%
\def\jurusan{Teknologi Multimedia Kreatif}




%-----------------------------------------------------------------------------%
% Informasi Dosen Pembimbing & Penguji
%-----------------------------------------------------------------------------%

% Tuliskan pembimbing
%
\def\pembimbingSatu{Zulhaydar Fairozal Akbar, S.ST., M.Sc.}
\def\pembimbingDua{Rizky Yuniar Hakkun, S.Kom., M.T.}
\def\pembimbingTiga{}

% Tuliskan NIP pembimbing
%
\def\nipPembimbingSatu{19890727202321132}
\def\nipPembimbingDua{198106222008121003}
\def\nipPembimbingTiga{}

% Tuliskan penguji
%
\def\pengujiSatu{Penguji Pertama Anda}
\def\pengujiDua{Penguji Kedua Anda}
\def\pengujiTiga{}
\def\pengujiEmpat{}
\def\pengujiLima{}
\def\pengujiEnam{}


% Tuliskan NIP penguji
%
\def\nipPengujiSatu{}
\def\nipPengujiDua{}
\def\nipPengujiTiga{}
\def\nipPengujiEmpat{}
\def\nipPengujiLima{}
\def\nipPengujiEnam{}




%-----------------------------------------------------------------------------%
% Informasi Lain (Asal Fakultas, Tanggal, dsb.)
%-----------------------------------------------------------------------------%

% Tuliskan tahun publikasi
\def\tahunPublikasi{2024}


% Tuliskan gelar yang akan diperoleh dengan menyerahkan laporan ini
%
\def\gelar{S.Tr.Kom.}


% Tuliskan tanggal pengesahan laporan, waktu dimana laporan diserahkan ke
% penguji/sekretariat
%
\def\tanggalSiapSidang{Tanggal Bulan Tahun}


% Tuliskan tanggal keputusan sidang dikeluarkan dan penulis dinyatakan
% lulus/tidak lulus
%
\def\tanggalLulus{Tanggal Bulan Tahun}

% Tuliskan informasi Kepala Jurusan
%
\def\kajur{Kholid Fathoni, S.Kom., M.Kom.}
\def\nipKajur{}


% Tuliskan informasi Ketua Program Studi
%
\def\kaprodi{Rizky Yuniar Hakkun, S.Kom., M.T.}
\def\nipKaprodi{198106222008121003}




%-----------------------------------------------------------------------------%
% Judul Setiap Bab
%-----------------------------------------------------------------------------%
% Berikut adalah judul-judul setiap bab.
% Silahkan diubah sesuai dengan kebutuhan.

\Var{\KataPengantar}{Kata Pengantar}
\Var{\babSatu}{Pendahuluan}
\Var{\babDua}{Tinjauan Pustaka}
\Var{\babTiga}{Perancangan Sistem}
\Var{\babEmpat}{Pengujian dan Analisa}
\Var{\kesimpulan}{Kesimpulan dan Saran}




%-----------------------------------------------------------------------------%
% Capitalized Variables
% Anda tidak perlu mengubah apapun di bagian ini
%-----------------------------------------------------------------------------%

\Var{\Judul}{\judul}
\Var{\Type}{\type}
\Var{\Penulis}{\penulis}
\Var{\Jenjang}{\jenjang}
\Var{\Prodi}{\prodi}
\Var{\Jurusan}{\jurusan}





% Daftar istilah yang mungkin perlu ditandai
%
%
% @author Atqa Munzir
% @version 1.0.0
%
% Mendaftar seluruh istilah yang mungkin akan perlu dijadikan
% italic atau bold pada setiap kemunculannya dalam dokumen.
%

\var{\license}{\f{Creative Common License 1.0 Generic}}
\var{\bslash}{$\setminus$}



\begin{document}


% Daftar pemenggalan suku kata dan istilah dalam LaTeX
%
\include{_internals/hypeindonesia}




%-----------------------------------------------------------------------------%
% Front
%-----------------------------------------------------------------------------%

% Sampul
%
%
% Sampul Laporan
%
% @author Atqa Munzir
% @version 1.0.0
% @edit by
%  -
%




\ThisULCornerWallPaper{1.0}{assets/backgrounds/blue.png}

\begin{titlepage}
  \centering
  \begin{tabular}{L{0.5\textwidth}T{0.5\textwidth}}
    \includegraphics[width=2.5cm]{assets/images/logo_pens.png} &
    \makebox[0.5\textwidth][c]{\raisebox{-2.75\height}{\large\bo{\Type}}}
  \end{tabular}

  \vspace*{2.5cm}

  \begin{adjustwidth}{1.0cm}{1.0cm}
    \centering
    \large\textcolor{white}{\bo{\Judul}} \\
  \end{adjustwidth}

  \vspace*{0.5cm}

  % penulis dan nrp
  \large
  \textcolor{white}{
    \f{Oleh:} \\ [5pt]
    \bo{\underline{\Penulis}} \\
    \bo{NRP. \nrp} \\
  }

  \vspace*{1.25cm}

  % dosen pembimbing dan nip
  \large
  \textcolor{white}{
    \f{Dosen Pembimbing:} \\ [5pt]
    \bo{\underline{\pembimbingSatu}} \\
    \bo{NIP. \nipPembimbingSatu} \\ [5pt]
    \bo{\underline{\pembimbingDua}} \\
    \bo{NIP. \nipPembimbingDua} \\
  }

  \vspace*{2.15cm}

  % informasi mengenai program studi
  \begin{adjustwidth}{-2cm}{-2cm}
    \centering
    \textcolor{white}{
      \bo{
        PROGRAM STUDI \Jenjang \ \Prodi \\
        JURUSAN \Jurusan \\
        POLITEKNIK ELEKTRONIKA NEGERI SURABAYA \\
        \tahunPublikasi
      }
    }
  \end{adjustwidth}
\end{titlepage}

\ifodd\thechapterpagecount\forceclearchapter\fi

\pagenumbering{roman}

% Judul Dalam
%
\addtocontents{toc}{\protect\addvspace{-10pt}}
%
% Lembar Judul Dalam
%
% @author Atqa Munzir
% @version 1.0.0
% @edit by
%  -
%




\begin{titlepage}
  \centering
  \begin{tabular}{L{0.5\textwidth}T{0.5\textwidth}}
    \includegraphics[width=2.5cm]{assets/images/logo_pens.png} &
    \makebox[0.5\textwidth][c]{\raisebox{-2.75\height}{\large\bo{\Type}}}
  \end{tabular}

  \vspace*{2.0cm}

  \begin{adjustwidth}{1.0cm}{1.0cm}
    \centering
    \large\bo{\Judul} \\
  \end{adjustwidth}

  \vspace*{1.25cm}

  % penulis dan nrp
  \large
  \f{Oleh:} \\ [5pt]
  \bo{\underline{\Penulis}} \\
  \bo{NRP. \nrp} \\

  \vspace*{1.25cm}

  % dosen pembimbing dan nip
  \large
  \f{Dosen Pembimbing:} \\ [5pt]
  \bo{\underline{\pembimbingSatu}} \\
  \bo{NIP. \nipPembimbingSatu} \\ [5pt]
  \bo{\underline{\pembimbingDua}} \\
  \bo{NIP. \nipPembimbingDua} \\

  \vspace*{1.9cm}

  % informasi mengenai program studi
  \begin{adjustwidth}{-2.0cm}{-2.0cm}
    \centering
    \bo{
      PROGRAM STUDI \Jenjang \ \Prodi \\
      JURUSAN \Jurusan \\
      POLITEKNIK ELEKTRONIKA NEGERI SURABAYA \\
      \tahunPublikasi
    }
  \end{adjustwidth}
\end{titlepage}

\ifodd\thechapterpagecount%
% Lembar Kosong
%
% @author Atqa Munzir
% @version 1.0.0
% @edit by
%	-
%




\begin{center}
    \vspace*{7cm}
    (Halaman ini sengaja dikosongkan)
\end{center}
\fi

\strcompare{Proyek Akhir}{\type} {\setcounter{page}{3}} {}


% Lembar Orisinalitas
%
\addtocontents{toc}{\protect\addvspace{-10pt}}
\addChapter{Lembar Orisinalitas}
%
% Lembar Pernyataan Orisinalitas
%
% @author  Atqa Munzir
% @version 1.0.0
% @edit by
%  -
%


%-----------------------------------------------------------------------------%
\chapter*{Lembar Pernyataan Orisinalitas}
%-----------------------------------------------------------------------------%


\vspace*{2.5cm}

% Untuk input gambar tanda tangan, silahkan sesuaikan xshift, yshift, dan width dengan gambar tanda tangan Anda
%\begin{tikzpicture}[remember picture,overlay,shift={(current page.north east)}]
%\node[anchor=north east,xshift=-8.5cm,yshift=-14.2cm]{\includegraphics[width=3cm]{assets/pics/tanda_tangan_wikipedia.png}};
%\end{tikzpicture}

{
  \centering
  Saya selaku penulis menyatakan bahwa \type\ ini adalah benar-benar hasil karya sendiri, dan semua sumber/referensi baik yang dikutip maupun dirujuk telah saya nyatakan dengan benar. \\
}

\vspace*{4.0cm}

{
  \centering
  Surabaya, \tanggalSiapSidang \\
  Penulis yang menyatakan, \\
}

\vspace*{1.5cm}

{
  \centering
  \penulis \\
  NRP. \nrp \\
}

\newpage

\ifodd\thechapterpagecount%
% Lembar Kosong
%
% @author Atqa Munzir
% @version 1.0.0
% @edit by
%	-
%




\begin{center}
    \vspace*{7cm}
    (Halaman ini sengaja dikosongkan)
\end{center}
\fi


% Lembar Pengesahan
%
\addChapter{Lembar Pengesahan}
%
% Lembar Pengesahan Proyek Akhir
%
% @author  Atqa Munzir
% @version 1.0.0
% @edit by
%  -
%


%-----------------------------------------------------------------------------%
\chapter*{Lembar Pengesahan Proyek Akhir}
%-----------------------------------------------------------------------------%


\ThisULCornerWallPaper{1.0}{assets/backgrounds/lembar_pengesahan.png}

{
  \centering
  "\MakeUppercase{\judul}"
}

\vspace*{10pt}

{
  \centering
  Oleh: \\
  \bo{\underline{\penulis}} \\
  NRP. \nrp \\
}

\vspace*{25pt}

{
  \centering
  \type \ ini diajukan sebagai salah satu syarat untuk \\
  Memperoleh Gelar Sarjana Terapan Komputer (S.Tr.Kom.) \\
  Program Studi \jenjang\ \prodi \\
  Jurusan \jurusan \\
  Politeknik Elektronika Negeri Surabaya \\
}

\vspace*{5pt}

{
  \centering
  Disetujui dan disahkan pada tanggal \tanggalSiapSidang\ oleh:
}

\vspace*{5pt}

\begin{adjustwidth}{-12.0cm}{-12.0cm}
  \centering
  \begin{tabular}{p{6.0cm} p{6.0cm}}
    \bo{Tim Penguji} & \bo{Dosen Pembimbing} \\
    1. & 1. \\
    \vspace*{0.75cm} & \\
    \bo{\underline{\pengujiSatu}} & \bo{\underline{\pembimbingSatu}} \\
    NIP. \nipPengujiSatu & NIP. \nipPembimbingSatu \\
    2. & 2. \\
    \vspace*{0.75cm} & \\
    \bo{\underline{\pengujiDua}} & \bo{\underline{\pembimbingDua}} \\
    NIP. \nipPengujiDua & NIP. \nipPembimbingDua \\
    \multicolumn{2}{c}{\vspace*{0.1cm}} \\
    \multicolumn{2}{c}{
      \begin{tabular}{c}
        Mengetahui:\\
        Ketua Program Studi \prodi \\ [1.25cm]
        \bo{\underline{\kaprodi}} \\
        NIP. \nipKaprodi
      \end{tabular}
    }
  \end{tabular}
\end{adjustwidth}

\newpage

% Lembar Pengesahan dari PDF lain
%\putpdf{assets/pdfs/pengesahanSidang.pdf}
\ifodd\thechapterpagecount%
% Lembar Kosong
%
% @author Atqa Munzir
% @version 1.0.0
% @edit by
%	-
%




\begin{center}
    \vspace*{7cm}
    (Halaman ini sengaja dikosongkan)
\end{center}
\fi


% Kata Pengantar
%
\addChapter{Kata Pengantar}
%
% Lembar Kata Pengantar
%
% @author Atqa Munzir
% @version 1.0.0
% @edit by
%  -
%


%-----------------------------------------------------------------------------%
\chapter*{\KataPengantar}
%-----------------------------------------------------------------------------%


{
  \hbadness=10000

  \hspace*{1.0cm}
  Puji syukur kehadirat Tuhan Yang Maha Esa atas segala rahmat dan karunia-Nya sehingga penulis dapat menyelesaikan laporan \type\ ini dengan baik. Laporan ini disusun sebagai salah satu syarat untuk memenuhi tugas akhir pada program studi \jenjang\ \prodi\ di Politeknik Elektronika Negeri Surabaya.

  Penulis menyadari bahwa dalam penyusunan laporan ini tidak lepas dari bantuan dan dukungan berbagai pihak. Oleh karena itu, pada kesempatan ini, penulis ingin menyampaikan ucapan terima kasih yang sebesar-besarnya kepada:

  \begin{enumerate}[topsep=0pt,itemsep=-1ex,partopsep=0ex,parsep=1ex]
    \item Allah SWT atas limpahan rahmat dan ridha-Nya sehingga penulis dapat menyelesaikan tugas akhir ini.
    \item Rasulullah Muhammad SAW sebagai suri tauladan penulis dan sebagai sumber inspirasi bagi penulis.
    \item Orang tua dan keluarga yang selalu memberikan dukungan, doa, dan bantuan selama pengerjaan tugas akhir ini.
    \item Bapak \kajur, selaku Kepala Jurusan \jurusan.
    \item Bapak \kaprodi, selaku Ketua Program Studi \prodi\ beserta dosen pembimbing dua selama tugas akhir.
    \item Bapak \pembimbingSatu, selaku dosen pembimbing satu selama tugas akhir.
    \item Seluruh pihak lain yang telah membantu kesuksesan tugas akhir ini, meskipun tidak dapat disebutkan satu persatu.
  \end{enumerate}

  Penulis menyadari bahwa laporan \type\ ini masih jauh dari sempurna. Oleh karena itu, apabila terdapat kesalahan atau kekurangan dalam laporan ini, Penulis memohon agar kritik dan saran bisa disampaikan langsung melalui \f{e-mail} \code{atqamz@gmail.com}.

  Akhir kata, semoga laporan tugas akhir ini dapat memberikan manfaat bagi semua pihak yang membacanya dan dapat menjadi referensi yang bermanfaat bagi peneliti-peneliti berikutnya.
}

% Untuk input gambar tanda tangan, silahkan sesuaikan xshift, yshift, dan width dengan gambar tanda tangan Anda
%\begin{tikzpicture}[remember picture,overlay,shift={(current page.north east)}]
%\node[anchor=north east,xshift=-3cm,yshift=-6.2cm]{\includegraphics[width=3cm]{assets/pics/tanda_tangan_wikipedia.png}};
%\end{tikzpicture}

\begin{flushright}
  Surabaya, \tanggalSiapSidang \\ [0.1cm]
  \vspace*{1.0cm}
  \penulis
\end{flushright}

\newpage

\ifodd\thechapterpagecount%
% Lembar Kosong
%
% @author Atqa Munzir
% @version 1.0.0
% @edit by
%	-
%




\begin{center}
    \vspace*{7cm}
    (Halaman ini sengaja dikosongkan)
\end{center}
\fi


% Abstrak
%
\addChapter{Abstrak}
%
% Lembar Abstrak
%
% @author Atqa Munzir
% @version 1.0.0
% @edit by
%	-
%


%-----------------------------------------------------------------------------%
\chapter*{ABSTRAK}
%-----------------------------------------------------------------------------%


\singlespacing

\vspace*{0.5cm}

\noindent Isi abstrak.

\vspace*{0.2cm}

\noindent Kata kunci: \f{Keyword} satu, kata kunci dua

\newpage

%
% Lembar Abstract
%
% @author Atqa Munzir
% @version 1.0.0
% @edit by
%  -
%


%-----------------------------------------------------------------------------%
\chapter*{ABSTRACT}
%-----------------------------------------------------------------------------%


\singlespacing

\vspace*{0.5cm}

\noindent Abstract content.

\vspace*{0.2cm}

\noindent Key words: Keyword one, keyword two

\newpage



% Daftar Isi, Gambar, dan Tabel
%
\addDefaultListPage{\tableofcontents}
\addDefaultListPage{\listoffigures}
\addDefaultListPage{\listoftables}


% Daftar Kode Program
% Comment to disable.
%
\addCustomListPage{\listoflistings}{\lstlistlistingname}


% Daftar Equation (Persamaan Matematis)
% Uncomment to use.
%
% \addCustomListPage{\listofequ}{\listofequname}


% Daftar Isi yang Didefinisikan Sendiri (Custom)
% Definisi jenis objek baru dapat dilakukan di pens.sty
% Uncomment to use.
%
% \addCustomListPage{\listofthing}{\listofthingname}


% Daftar Lampiran
% Comment to disable.
%
\addCustomListPage{\listofappendix}{\listofappendixname}


\noCAPinToC % Revert to original \addcontentsline formatting
\ifodd\thechapterpagecount\clearpage\else\forceclearchapter\fi


% Jika penomoran romawi selesai di ganjil
%
%\naiveoddclearchapter

% Jika penomoran romawi selesai di genap
%
%\naiveevenclearchapter




%-----------------------------------------------------------------------------%
% BODY
%-----------------------------------------------------------------------------%

\pagenumbering{arabic}

\setoddevenheader
%-----------------------------------------------------------------------------%
\chapter{\babSatu}
\label{bab:1}
%-----------------------------------------------------------------------------%
Bab ini berisi tentang pendahuluan yang terdiri dari latar belakang, rumusan masalah, batasan masalah, tujuan, metodologi, serta sistematika penulisan dari penelitian \cite{book:sample}.

Latar Belakang

Pada era digital saat ini, Game analytics telah menjadi elemen krusial dalam pengembangan game. Game analytics merupakan proses pengumpulan, analisis, dan interpretasi data dari permainan yang bertujuan untuk meningkatkan pengalaman pengguna dan efektivitas bisnis. Pengembangan game tidak lagi hanya bergantung pada intuisi dan pengalaman pengembang, tetapi juga pada data yang dikumpulkan dari interaksi pengguna dengan game tersebut. Penggunaan game analytics memungkinkan pengembang untuk memahami perilaku pemain, mengidentifikasi masalah desain, dan mengevaluasi kinerja game secara menyeluruh. Hal ini tidak hanya meningkatkan kualitas game tetapi juga mendukung keputusan bisnis yang lebih baik melalui wawasan berbasis data.

Layanan berbasis cloud memiliki beberapa kekurangan signifikan, terutama dalam hal ketergantungan pada server cloud yang harus selalu aktif agar layanan dapat digunakan. Sebagai contoh, layanan game analytics yang dikembangkan oleh Sakti (2018) tidak dapat diakses oleh pengguna kecuali layanan cloud-nya diaktifkan kembali. Hal ini dapat menghambat pengembang game yang bergantung pada layanan tersebut [2]. Sementara layanan berbasis on-premises dapat diaktifkan dan digunakan oleh pengguna secara langsung. Membuat pengguna bisa memiliki kontrol penuh terhadap layanan yang digunakan [3].

Masalah signifikan lainnya dari layanan berbasis cloud adalah biaya. Layanan analytics ini biasanya berbayar dan cenderung mahal. Contohnya biaya layanan GameAnalytics Pro seharga USD 299/ bulan [4]. Biaya ini dapat menjadi penghalang bagi pengembang kecil atau perusahaan dengan anggaran terbatas, yang mungkin tidak mampu membayar layanan premium. Dampaknya, pengembang kecil dapat kesulitan untuk bersaing dengan perusahaan yang lebih besar yang memiliki sumber daya lebih besar.

Karena layanan game analytics yang ada saat ini selalu berbasis cloud, pengembang tidak dapat mengetahui data-data yang diolah oleh layanan game analytics. Kurangnya transparansi ini dapat menyebabkan masalah kepercayaan dan kesulitan dalam memecahkan masalah. Sebagai alternatif, layanan yang bersifat open source memberikan akses penuh kepada pengguna terhadap source code, sehingga mempermudah dalam hal debugging, peningkatan, dan penyesuaian fitur sesuai kebutuhan [5]. Selain itu, karena tidak ada biaya lisensi yang harus dibayar, solusi open source sering kali lebih ekonomis dalam jangka panjang, terutama bagi perusahaan kecil dan menengah yang memiliki anggaran terbatas.

Rumusan Masalah

Proyek akhir ini berfokus pada pengembangan on-premises analytics game service yang bersifat open source. Berdasarkan latar belakang masalah yang telah dijelaskan, maka rumusan masalah dalam penelitian ini adalah sebagai berikut:

Bagaimana merancang dan membangun kerangka arsitektur backend yang dapat diimplementasikan secara on-premises untuk pembuatan game service?

Bagaimana membangun kerangka arsitektur client SDK untuk game service yang efisien dan mudah diintegrasikan ke backend menggunakan Unity plugin?

Bagaimana cara merancang dan mengimplementasikan fitur analytics berupa A/B Testing, D/MAU (Daily/Monthly Active User) tracker, UA (User Acquisition) tracker, Custom Event tracker, dan User Playtime Session tracker yang terintegrasi penuh dengan game service secara client dan backend?

Bagaimana membuat struktur dokumentasi praktis dan dokumentasi pengguna akhir untuk membantu pengembang dan pengguna memahami layanan yang dikembangkan?

Batasan Masalah

Pada proyek akhir ini, terdapat beberapa batasan masalah yang ditetapkan untuk memperjelas ruang lingkup dan fokus dari penelitian yang dilakukan. Batasan masalah tersebut adalah sebagai berikut:

Proyek ini tidak mencakup pembuatan atau pengembangan model bisnis untuk produk yang dikembangkan. Fokus utama adalah pada aspek teknis dan implementasi layanan.

Proyek ini menyediakan backend game service dengan fitur analytics. Dan frontend yang terbagi menjadi 2, yakni:

Client SDK berbasis Unity plugin.

Client Dashboard berbasis web.

Semua komponen ini bersifat open source dan diimplementasikan secara on-premises

Proyek ini mencakup pembuatan dokumentasi praktis yang ditujukan untuk pengembang, serta dokumentasi pengguna akhir yang membantu pengguna memahami dan menggunakan layanan yang dikembangkan.

Tujuan dan Manfaat

Bagian ini menjelaskan tujuan yang ingin dicapai dan manfaat yang diharapkan dari pelaksanaan proyek akhir ini. Tujuan merupakan hasil yang ingin dicapai melalui proses penelitian dan pengembangan yang dilakukan, sementara manfaat mencakup kontribusi penelitian ini baik dari sisi teoritis maupun praktis.

Tujuan

Proyek akhir ini memiliki beberapa tujuan yang ingin dicapai, yaitu:

Mengembangkan fitur analytics yang dapat diintegrasikan sepenuhnya pada game service berbasis open source dan on-premises.

Mengatasi masalah yang dihadapi oleh game service saat ini seperti kurangnya kontrol, transparansi, dan keterbatasan kustomisasi dengan menyediakan solusi yang lebih fleksibel dan memungkinkan pengguna untuk memiliki kendali langsung atas data mereka.

Berkontribusi pada kebutuhan analisis pasar dalam industri game dengan menyediakan data analytics yang komprehensif dan dapat diakses oleh pengembang game dan publik.

Manfaat

Penelitian ini diharapkan dapat memberikan beberapa manfaat, baik secara teoritis maupun praktis, sebagai berikut:

Manfaat Teoritis

Menambah wawasan dan pengetahuan dalam bidang game analytics dan pengembangan game serivice yang bersifat open source dan on-premises.

Mengembangkan model pengelolaan data yang dapat diintegrasikan dengan game service sehingga dapat dijadikan referensi bagi penelitian selanjutnya.

Manfaat Praktis

Meningkatkan kualitas game service dengan menyediakan fitur analytics yang lebih terjangkau dan dapat disesuaikan sesuai kebutuhan pengguna.

Menghemat biaya pengelolaan data dan sumber daya melalui penerapan solusi open source dan on-premises yang lebih efisien.

Memberikan transparansi yang lebih baik dan kontrol langsung atas data bagi pengguna, sehingga mereka dapat menyesuaikan layanan sesuai dengan kebutuhan spesifik mereka.

Membantu pengembang game dalam melakukan analisis pasar dengan menyediakan data analytics yang dapat digunakan untuk mengidentifikasi tren dan kebutuhan pasar.

Metodologi Penelitian

Proyek akhir ini menggunakan pendekatan penelitian yang terstruktur untuk mencapai tujuan yang telah ditetapkan. Metodologi penelitian yang diterapkan mencakup beberapa tahapan sebagai berikut:

Gambar 1.1 menunjukkan alur metodologi dari pengerjaan proyek akhir ini. Selanjutnya akan dibahas dari setiap alur yang telah disebutkan.

Studi Literatur

Pada studi leteratur dilakukan pencarian materi dan referensi penunjang terkait topik proyek akhir yang diangkat. Referensi didapat dari buku, thesis, jurnal, penelitian serupa, dan penelitian terdahulu. Studi literatur bertujuan untuk mencari data dan menganalisa penelitian yang telah dikumpulkan sehingga dapat digunakan dalam penelitian.

Perancangan Sistem

Pada tahap ini, dilakukan berbagai kegiatan yang berfokus pada identifikasi masalah dan persiapan teknis yang diperlukan untuk pengembangan sistem. Perancangan sistem bertujuan untuk menciptakan sebuah blueprint atau peta jalan yang jelas mengenai bagaimana sistem akan dikembangkan, diintegrasikan, dan diujicobakan. Berikut adalah rincian dari setiap langkah dalam tahap perancangan sistem:

Identifikasi Masalah

Pada identifikasi masalah, yaitu melakukan penelusuran masalah dari literatur yang bersumber dari buku, jurnal, maupun penelitian. Tahap ini bertujuan untuk menentukan masalah dan topik yang akan dibahas dalam proyek akhir.

Persiapan Teknis

Pada tahap ini, dilakukan persiapan teknis yang meliputi pemilihan perangkat lunak, framework, dan perangkat keras yang akan digunakan dalam pengembangan sistem. Ini termasuk pemilihan teknologi yang digunakan untuk pengerjaan proyek akhir.

Pengembangan Sistem

Tahap pengembangan sistem merupakan inti dari proyek akhir ini. Pada tahap ini, semua hasil analisis dan perancangan yang telah dilakukan sebelumnya diimplementasikan menjadi sebuah sistem yang nyata. Pengembangan sistem mencakup beberapa langkah penting yang harus dilaksanakan secara berurutan dan terstruktur. Berikut adalah penjelasan dari setiap langkah dalam pengembangan sistem:

Perancangan Arsitektur dan Desain

Pada tahap ini, dilakukan perancangan arsitektur sistem dan desain antarmuka pengguna. Arsitektur sistem mencakup desain backend, frontend dan bagaimana keduanya berinteraksi.

Implementasi

Implementasi sistem dilakukan dengan menggunakan teknologi yang telah dipilih. Backend dan frontend akan dikembangkan berdasarkan arsitektur yang telah dirancang. Tahap ini mencakup pengkodean, pengujian unit, dan dokumentasi kode.

Integrasi

Tahap ini melibatkan integrasi antara berbagai komponen sistem, termasuk integrasi antara frontend dan backend, serta integrasi dengan database atau layanan penyimpanan data lainnya.

Pengujian Lokal

Pengujian dilakukan pada tahap ini untuk memastikan bahwa setiap komponen sistem berfungsi dengan baik sebelum diimplementasikan secara keseluruhan. Pengujian lokal mencakup pengujian unit, pengujian integrasi, dan pengujian fungsionalitas.

Dokumentasi Pengembangan

Dokumentasi pengembangan mencakup penulisan dokumentasi teknis dan dokumentasi pengguna akhir. Dokumentasi teknis ditujukan untuk pengembang dan mencakup detail implementasi, sedangkan dokumentasi pengguna akhir membantu pengguna memahami dan menggunakan sistem.

Pengujian dan Analisa Hasil

Setelah pengembangan sistem selesai, dilakukan pengujian dan analisa hasil untuk memastikan sistem berfungsi sesuai dengan spesifikasi yang telah ditentukan. Pengujian dilakukan pada game yang telah dibuat oleh game studio. Pengujian dinyatakan berhasil apabila plugin dapat diimplementasikan pada game dan dapat melakukan komunikasi ke backend game service. Selain itu, pengujian juga dianggap berhasil apabila data yang direkam melalui game dapat disimpan pada database backend dan disajikan dalam bentuk informasi yang berguna di client dashboard.

Penyusunan Laporan

Tahap ini merupakan tahap dokumentasi dari semua tahapan proses di atas. Dokumentasi tersebut disusun dalam bentuk laporan yang berisi tentang dasar teori, metode yang digunakan, serta hasil yang diperoleh selama pengerjaan proyek akhir.

Sistematika Penulisan

Proposal proyek akhir ini disusun dalam empat bab yang terstruktur untuk memberikan gambaran yang jelas dan komprehensif mengenai penelitian yang dilakukan. Sistematika penulisan dalam proposal proyek akhir ini adalah sebagai berikut:

BAB I: PENDAHULUAN

Bab ini berisi latar belakang masalah, rumusan masalah, tujuan penelitian, batasan masalah, dan manfaat penelitian. Pada bab ini, dijelaskan alasan mengapa penelitian ini penting dilakukan dan apa yang ingin dicapai melalui penelitian ini.

BAB II TINJAUAN PUSTAKA

Bab ini berisi kajian teori dan literatur yang relevan dengan penelitian yang dilakukan. Pada bab ini, dijelaskan teori-teori yang mendasari penelitian serta hasil-hasil penelitian terdahulu yang berkaitan dengan topik yang diteliti.

BAB III: PENGEMBANGAN SISTEM

Bab ini menjelaskan tahapan-tahapan dalam pengembangan sistem yang dilakukan, termasuk metodologi penelitian, desain sistem, implementasi, dan pengujian.

BAB IV: LINIMASA PENELITIAN

Bab ini berisi linimasa atau jadwal penelitian yang menggambarkan tahapan-tahapan penelitian yang telah dilakukan beserta waktu pelaksanaannya.

\clearchapter
%-----------------------------------------------------------------------------%
\chapter{\babDua}
\label{bab:2}
%-----------------------------------------------------------------------------%
Penelitian Terkait

Bagian ini membahas beberapa penelitian terkait yang menjadi landasan dan referensi dalam pengembangan proyek akhir ini. Dengan mengkaji penelitian-penelitian sebelumnya, kita dapat memahami perkembangan terkini dalam bidang yang sama dan mengidentifikasi celah yang perlu diisi oleh penelitian ini. Penelitian terkait memberikan konteks yang lebih luas dan membantu mengarahkan penelitian agar lebih relevan dan bermanfaat.

Pengembangan Perangkat Lunak untuk Membantu Evaluasi User Experience dalam Game dengan Metode Heatmaps

Penelitian ini berfokus pada pengembangan perangkat lunak yang bertujuan untuk membantu analisis pengalaman pengguna (user experience) dalam game melalui metode heatmaps. Heatmaps merupakan representasi grafis dari data di mana nilai-nilai individu dalam sebuah matriks direpresentasikan sebagai warna. Dalam konteks penelitian ini, heatmaps digunakan untuk melacak perilaku pengguna selama berinteraksi dengan game, seperti pergerakan cursor, posisi klik, dan pandangan mata.

Metode yang digunakan dalam penelitian ini melibatkan beberapa langkah. Pertama, dilakukan studi literatur untuk memahami teknologi dan metode yang relevan. Selanjutnya, dilakukan perancangan sistem yang mencakup pembuatan antarmuka pengguna, integrasi dengan aplikasi pelacakan mata (GazePointer), serta implementasi fitur perekaman layar dan visualisasi heatmaps.

Pengujian sistem dilakukan dengan menggunakan game “Aquaculture Land” sebagai studi kasus. Data yang dihasilkan mencakup heatmaps yang menunjukkan area dominan yang dilihat dan diklik oleh pengguna. Hasil analisis data ini diharapkan dapat membantu pengembang game dalam meningkatkan kualitas user experience secara keseluruhan.

Penelitian ini menunjukkan bahwa perangkat lunak yang dikembangkan dapat membantu pengembang dalam mengevaluasi user experience secara efektif. Program ini mudah digunakan dalam proses pengambilan dan visualisasi data, serta dapat memberikan wawasan yang berharga untuk meningkatkan desain antarmuka game [6].

Layanan Analytics Terbuka Sebagai Solusi Percepatan Pertumbuhan Game Indie Lokal Untuk Prediksi Scaling dan Pemerataan Pasar Potensial

Penelitian ini bertujuan untuk mengembangkan layanan analytics terbuka yang dirancang untuk mempercepat pertumbuhan game indie lokal dengan memprediksi scaling dan pemerataan pasar potensial. Di Indonesia, meskipun banyak studio game indie yang bermunculan, mereka sering kali mengabaikan pentingnya layanan analytics dalam game mereka untuk memprediksi pasar potensial. Hal ini disebabkan kurangnya layanan analytics yang terbuka dalam komunitas developer indie, di mana mayoritas layanan hanya menyediakan informasi kepada satu akun developer.

Metode yang digunakan dalam penelitian ini mencakup beberapa langkah. Pertama, dilakukan studi literatur untuk memahami teknologi dan metode yang relevan. Selanjutnya, dilakukan perancangan sistem yang mencakup pembuatan REST API, dokumentasi API, serta halaman penyajian data berupa statistik deskriptif dengan informasi Game Playtime Session, D/MAU (Daily/Monthly Active User), dan Device Retention.

Pengujian sistem dilakukan dengan menggunakan game “Ninja Adventure” dari studio indie Vizard Tales sebagai studi kasus. Data yang dihasilkan mencakup informasi mengenai aktivitas pemain yang direkam melalui package dan plugin yang diimplementasikan dalam game engine Unity3D. Hasil analisis data ini diharapkan dapat membantu developer game dalam meningkatkan kualitas dan efektivitas game mereka dengan memanfaatkan informasi yang dibagikan secara terbuka dalam komunitas.

Penelitian ini menunjukkan bahwa sistem analytics yang dikembangkan dapat membantu developer game indie dalam menganalisis aktivitas pemain secara efektif dan berbagi informasi untuk prediksi scaling serta pemerataan pasar potensial. Sistem ini mudah diimplementasikan dan memberikan wawasan yang berharga untuk pengembangan game lebih lanjut [2].

Data Cracker: Developing a Visual Game Analytic Tool for Analyzing Online Gameplay

Penelitian ini membahas pengembangan alat analisis game visual yang dirancang untuk memantau perilaku pemain selama memainkan sesi game online. Alat yang dinamakan Data Cracker ini dikembangkan untuk memonitor gameplay di “Dead Space 2,” sebuah game dari franchise Dead Space. Tujuan utama dari alat ini adalah meningkatkan literasi data tim game dengan cara melibatkan seluruh anggota tim dalam analisis game.

Data Cracker dibangun dengan pendekatan yang berfokus pada visualisasi data melalui berbagai prototipe awal dan branding alat untuk tim Dead Space 2. Alat ini memungkinkan tim untuk memantau berbagai metrik pemain, seperti waktu mulai dan selesai permainan, posisi pemain, jenis senjata yang digunakan, dan lain-lain. Data tersebut dikumpulkan melalui sistem telemetri yang terintegrasi dalam kode game, mengirimkan data ke server untuk kemudian dianalisis dan divisualisasikan pada antarmuka web klien.

Penelitian ini menunjukkan bahwa alat analisis game visual seperti Data Cracker dapat meningkatkan proses desain game dengan memberikan wawasan yang lebih dalam tentang perilaku pemain. Alat ini tidak hanya membantu dalam pengembangan game tetapi juga dalam memelihara dan meningkatkan kualitas game setelah dirilis karena terus memantau dan menganalisis data pemain [8].

Produk Terkait

Bagian ini membahas produk-produk terkait yang memiliki fungsi atau fitur serupa dengan proyek akhir yang sedang dikembangkan. Tujuan dari peninjauan ini adalah untuk memahami kekuatan dan kelemahan produk-produk yang ada, serta untuk mendapatkan inspirasi dan wawasan yang dapat diterapkan dalam pengembangan proyek akhir.

AccelByte

AccelByte adalah platform layanan backend untuk game yang menyediakan berbagai fitur termasuk analytics, manajemen akun, dan penyimpanan data [9]. Komponen analytics yang ditawarkan, AccelByte Metrics, memungkinkan pengembang game untuk mengumpulkan, menyimpan, memproses, dan mengekspor data performa game. Beberapa metrik penting yang disediakan meliputi Daily Active Users (DAU), Monthly Active Users (MAU), retensi pemain, sesi permainan, dan monetisasi.

IronSource

IronSource adalah platform monetisasi dan distribusi aplikasi yang menyediakan solusi komprehensif untuk mengoptimalkan pendapatan dan meningkatkan pertumbuhan pengguna [10]. IronSource Analytics menawarkan pelacakan dan pengukuran kampanye iklan, analytics performa aplikasi, dan pencegahan kecurangan. Platform ini memungkinkan pengembang untuk melacak metrik penting seperti DAU, MAU, sesi permainan, dan retensi pemain. IronSource juga menyediakan alat A/B testing untuk membantu pengembang mengoptimalkan pengalaman pengguna dan meningkatkan pendapatan.

Adjust

Adjust adalah platform analytics yang fokus pada tracking dan pengukuran ads campaign aplikasi mobile [11]. Adjust Analytics menyediakan layanan untuk install/uninstall tracking, application performance analytics, dan fraud prevention. Adjust juga menawarkan fitur custom event analytics dan D/MAU.

Komparasi Produk

Komparasi ini bertujuan untuk melihat bagaimana masing-masing produk terkait beroperasi dalam hal fitur yang ditawarkan, biaya, dan model layanan.

Tabel 2.1. Tabel Komparasi Produk Terkait

Dengan membandingkan fitur dan kelebihan masing-masing produk, proyek akhir ini bertujuan untuk mengintegrasikan keunggulan dari ketiga platform ini, namun dengan menyediakan semua fitur utama secara gratis, sehingga dapat diakses oleh pengembang game dan publik tanpa beban biaya tambahan.

Teori Penunjang

Bagian ini membahas teori-teori yang mendasari penelitian dan pengembangan proyek akhir ini. Teori-teori ini memberikan landasan ilmiah dan metodologis yang kuat, serta membantu dalam memahami konteks dan penerapan berbagai konsep yang digunakan dalam proyek ini.

Game Analytics

Game analytics adalah proses mengidentifikasi dan menyampaikan pola-pola penting yang dapat digunakan sebagai dasar untuk membuat keputusan strategis yang lebih baik dalam pengembangan dan manajemen permainan [12]. Tujuan dari game analytics adalah untuk memecahkan masalah, membuat prediksi dalam bisnis game, membantu pengambilan keputusan, mempromosikan tindakan optimasi, dan meningkatkan kinerja bisnis permainan.

Gambar 2.1. Pecahan Game Analytics

Seperti pada gambar di atas, Game Analytics terbagi menjadi dua jenis, yaitu Game Value Chain dan Orthogonal to Value Chain.

Game Value Chain

Game Player Analytics

Melibatkan analisis terhadap pemain game, termasuk segmentasi pemain, perilaku pemain, gameplay, antarmuka, sistem, proses, dan performa.

Game Development Analytics

Fokus pada analisis selama pengembangan game, termasuk sistem yang digunakan, proses pengembangan, dan performa game selama tahap pengembangan.

Game Publishing Analytics

Menganalisis aspek penerbitan game, termasuk akuisisi pemain, retensi pemain, dan pendapatan yang dihasilkan dari game.

Distribution Channel Analytics

Menganalisis saluran distribusi game untuk memahami bagaimana game didistribusikan dan diterima oleh pasar.

Orthogonal to Value Chain

Game Prediction

Meliputi prediksi seperti churn prediction (prediksi pemain yang akan berhenti bermain) dan revenue prediction (prediksi pendapatan).

Data Visualization

Fokus pada visualisasi data untuk memudahkan pemahaman dan interpretasi data analytics yang telah dikumpulkan.

Fitur Analytics yang akan dikerjakan

A/B Testing

A/B testing memungkinkan pengembang membuat keputusan berdasarkan data untuk mengoptimalkan aplikasi mereka. Proses ini melibatkan pembandingan dua varian (A dan B) dengan pengguna yang ditugaskan secara acak untuk menghindari bias. Tujuan eksperimen harus ditentukan dengan jelas untuk mengukur KPI yang relevan, seperti retensi pemain dan tingkat konversi. Analisis hasil A/B testing membantu memahami dampak dari perubahan yang diuji terhadap metrik kinerja utama dan meningkatkan pengalaman pengguna secara keseluruhan [13].

Daily/Monthly Active User Tracking

Daily Active User (DAU) dan Monthly Active User (MAU) adalah metrik penting dalam game analytics yang digunakan untuk mengukur jumlah pengguna unik yang berinteraksi dengan game dalam periode harian dan bulanan. Metrik ini memberikan wawasan tentang keterlibatan pemain dan popularitas game, yang sangat penting untuk strategi pengembangan dan pemasaran game [8].

Custom Event Tracking

Custom event tracking adalah teknik dalam game analytics yang memungkinkan pengembang untuk melacak interaksi spesifik pengguna dengan berbagai elemen dalam game yang tidak tercakup oleh metrik standar. Dengan menggunakan custom event tracking, pengembang dapat mengumpulkan data yang lebih terperinci tentang perilaku pemain dan mengidentifikasi area untuk perbaikan dan optimasi [14].

User Acquisition Tracking

User Acquisition (UA) tracking adalah proses mengukur efektivitas kampanye pemasaran dengan melacak asal pengguna baru yang berinteraksi dengan game. Metrik ini sangat penting untuk memahami dari mana pemain berasal, platform mana yang paling efektif, dan strategi pemasaran apa yang memberikan hasil terbaik [15].

User Playtime Session Tracking

User Playtime Session Tracking adalah proses pengumpulan dan analisis data terkait durasi dan frekuensi sesi bermain pengguna dalam sebuah game. Metrik ini sangat penting untuk memahami bagaimana pemain berinteraksi dengan game dan untuk mengidentifikasi pola bermain yang dapat digunakan untuk meningkatkan desain game dan pengalaman pengguna [15].

Client-Server Architecture

Client-Server Architecture adalah sebuah model arsitektur yang memisahkan tugas antara penyedia layanan (server) dan pemohon layanan (client) [16]. Server bertanggung jawab untuk menyimpan, memproses, dan mengelola data atau layanan, sementara klien mengirimkan permintaan ke server dan menampilkan hasil kepada pengguna. Arsitektur ini memungkinkan distribusi beban kerja yang lebih efisien, pemeliharaan yang lebih mudah, dan skalabilitas yang lebih baik dalam sistem jaringan modern.

Open Source Software

Open Source Software didefenisikan dari model lisensi yang memberikan pengguna kebebasan untuk menjalankan program untuk tujuan apapun, mempelajari dan memodifikasi source code, serta mendistribusikan kembali versi asli maupun yang telah dimodifikasi dari perangkat lunak tersebut. Prinsip-prinsip ini memastikan bahwa perangkat lunak tidak dibatasi oleh bidang penggunaan, wilayah, atau pasar, sehingga mendorong kolaborasi dan transparansi dalam pengembangan [5].

On-Premises Software

On-Premises Software mengacu pada model penerapan perangkat lunak di mana perangkat lunak diinstal dan dijalankan pada komputer atau server yang berada di lokasi fisik [3].

Software Development Kit (SDK)

Software Development Kit (SDK) adalah kumpulan alat pengembangan perangkat lunak yang disediakan dalam satu paket untuk membantu pembuatan aplikasi pada platform tertentu [17]. SDK mempermudah proses pengembangan dengan menyediakan komponen yang sudah jadi dan panduan menggunakannya.

Unity Plugin

Unity Plugin adalah sekumpulan kode yang dibuat di luar Unity dan digunakan untuk menambahkan fungsionalitas pada proyek Unity. Terdapat dua jenis plugin di Unity: Managed Plugins dan Native Plugins. Managed Plugins adalah assembly .NET yang dikelola dan dibuat dengan alat seperti Visual Studio, sedangkan Native Plugins adalah perpustakaan kode asli yang spesifik untuk platform tertentu. Plugin memungkinkan akses ke fitur sistem operasi dan perpustakaan pihak ketiga yang biasanya tidak dapat diakses langsung oleh Unity [18].

\clearchapter
%-----------------------------------------------------------------------------%
\chapter{\babTiga}
\label{bab:3}
%-----------------------------------------------------------------------------%
Analisis Kebutuhan

Analisis kebutuhan ini bertujuan untuk mengidentifikasi dan mendefinisikan kebutuhan yang harus dipenuhi oleh sistem yang dikembangkan dalam proyek akhir ini. Berdasarkan analisis terhadap penelitian-penelitian sebelumnya, ditemukan beberapa kekurangan yang harus diatasi dalam pengembangan layanan game analytics berbasis on-premises dan bersifat open source.

Kebutuhan Fungsional

Kebutuhan fungsional merujuk pada fitur dan fungsi yang harus disediakan oleh sistem agar dapat memenuhi tujuan dan kebutuhan pengguna.

Layanan Analytics On-Premises

Sistem harus menyediakan layanan analytics yang dapat diinstal dan dijalankan secara lokal (on-premises) di server pengguna. Hal ini mengatasi masalah keberlanjutan layanan berbasis cloud yang dapat mati atau tidak dapat diakses lagi.

Fleksibilitas Integrasi

Sistem harus memungkinkan integrasi dengan berbagai game tanpa memerlukan pengembangan alat tambahan yang kompleks atau pemasangan perangkat keras khusus pada device pemain.

Open Source

Sistem harus bersifat open source, memungkinkan pengguna untuk memodifikasi dan menyesuaikan layanan sesuai kebutuhan spesifik mereka. Ini memberikan fleksibilitas dan transparansi yang lebih besar, mengatasi masalah kontrol dan transparansi yang ditemukan dalam layanan analytics berbayar.

Kebutuhan non-Fungsional

Kebutuhan non-fungsional merujuk pada atribut dan karakteristik sistem yang tidak terkait langsung dengan fungsionalitas tetapi penting untuk performa dan pengalaman pengguna.

Kemudahan Instalasi dan Konfigurasi

Sistem harus mudah diinstal dan dikonfigurasi oleh pengguna, tanpa memerlukan pengetahuan teknis yang mendalam. Ini penting untuk meningkatkan adopsi sistem oleh pengembang game indie dan kecil.

Dukungan dan Dokumentasi

Sistem harus menyediakan dokumentasi yang lengkap dan jelas. Ini membantu pengguna dalam memecahkan masalah dan mengoptimalkan penggunaan sistem.

Dengan memenuhi kebutuhan fungsional dan non-fungsional ini, sistem yang dikembangkan diharapkan dapat memberikan solusi analytics yang handal, fleksibel, dan dapat diakses oleh berbagai pengembang game.

Perancangan Sistem

Bagian ini menjelaskan perancangan sistem yang dikembangkan dalam proyek akhir ini. Sistem dirancang untuk menyediakan layanan game analytics berbasis open source on-premises yang mudah diakses dan diintegrasikan oleh pengembang game. Gambar 3.1 menunjukkan diagram sistem yang diusulkan.

Gambar 3.1. Diagram Sistem

Komponen Sistem

Sistem ini terdiri dari beberapa komponen utama, yaitu:

Database

Berfungsi sebagai tempat penyimpanan data. Database ini dapat di-query untuk mendapatkan data yang dibutuhkan.

Backend Game Service

Backend Game Service adalah inti dari sistem ini. Layanan ini bertanggung jawab untuk menerima dan memproses data yang dikirim dari game melalui Unity SDK. Backend juga menyediakan fitur analitik yang memungkinkan pengembang untuk melakukan query dan mendapatkan data yang relevan dari database.

Analytics Feature

Merupakan bagian dari Backend Game Service yang menyediakan berbagai fungsi analytics. Data yang diproses oleh fitur analytics ini kemudian dapat disajikan melalui Frontend Dashboard.

Unity SDK

Unity SDK adalah kit pengembangan perangkat lunak yang disediakan untuk pengembang game yang menggunakan Unity. SDK ini memungkinkan game untuk berkomunikasi dengan Backend Game Service, mengirimkan data gameplay, dan menerima respon dari server.

Frontend Dashboard

Frontend Dashboard adalah antarmuka pengguna yang memungkinkan pengembang game untuk melihat dan menganalisis data yang dikumpulkan. Dashboard ini menyediakan berbagai visualisasi data dan alat analitik yang membantu pengembang memahami perilaku pemain dan mengoptimalkan desain game mereka.

Client

Client adalah perangkat yang digunakan oleh pemain untuk berinteraksi dengan game. Client mengirimkan input pengguna ke game dan menerima render dari gameplay. Data dari interaksi ini kemudian dikirim ke Backend Game Service untuk dianalisis.

Alur Proses

Sistem ini memiliki beberapa proses, yakni sebagai berikut:

Data Collection

Saat pemain berinteraksi dengan game, data gameplay dan input pengguna dikumpulkan oleh Unity SDK dan dikirim ke Backend Game Service.

Data Processing

Backend Game Service menerima data dari game dan menyimpannya dalam database. Data ini kemudian diproses oleh Analytics Feature untuk menghasilkan metrik analitik yang relevan.

Data Query and Response

Pengembang game dapat melakukan query ke database melalui Frontend Dashboard untuk mendapatkan data yang mereka butuhkan. Backend Game Service memproses query ini dan mengirimkan data yang diminta ke Frontend Dashboard.

Data Visualization

Frontend Dashboard menampilkan data yang telah diproses dalam bentuk visualisasi yang mudah dipahami. Pengembang game dapat menggunakan informasi ini untuk membuat keputusan yang lebih baik mengenai desain dan pengembangan game mereka.

Client Interaction

Selama permainan, client terus berinteraksi dengan game, mengirimkan input pengguna dan menerima render gameplay. Data dari interaksi ini terus dikirim ke Backend Game Service untuk dianalisis secara real-time.

Pengembangan Sistem

Bagian ini menjelaskan tahapan-tahapan dalam pengembangan sistem, mulai dari perancangan kerangka arsitektur hingga integrasi dan pengembangan fitur.

Pengembangan Kerangka Arsitektur Backend Game Service

Pengembangan kerangka arsitektur Backend Game Service melibatkan perancangan dan implementasi komponen backend yang menjadi inti dari layanan analytics.

Gambar 3.2. Rancangan Arsitektur Backend Game Service

Pengembangan Kerangka Arsitektur Game Service Client Unity SDK

Pengembangan kerangka arsitektur Unity SDK melibatkan pembuatan Software Development Kit (SDK) yang memungkinkan game yang dikembangkan di Unity untuk berkomunikasi dengan Backend Game Service. SDK ini menyediakan fungsi-fungsi yang memudahkan integrasi, seperti mengirim data gameplay, menangkap event penting dalam game, dan menerima respon dari server. SDK ini dirancang agar mudah digunakan oleh pengembang Unity, dengan dokumentasi lengkap dan contoh implementasi yang membantu pengembang memahami cara mengintegrasikan layanan backend ke dalam game mereka.

Gambar 3.3. Rancangan Arsitektur Game Service Client Unity SDK

Integrasi Backend Game Service dengan Client Unity SDK

Integrasi ini memastikan bahwa data yang dikumpulkan oleh game dapat dikirim ke Backend Game Service dan diproses dengan benar.

Gambar 3.4. Rancangan Integrasi Backend Game Service dengan Client Unity SDK

Pengembangan Fitur Analytics pada Game Service

Pengembangan fitur analytics melibatkan pembuatan fungsi-fungsi analytics yang dapat digunakan untuk menganalisis data game.

Gambar 3.5. Rancangan Fitur Analytics pada Game Service

Fitur ini mencakup pengumpulan data, pemrosesan data untuk menemukan pola atau anomali, dan visualisasi data yang membantu pengembang game memahami perilaku pemain.

Pengembangan Game Service Frontend Dashboard

Pengembangan Frontend Dashboard melibatkan pembuatan antarmuka pengguna yang memungkinkan pengembang game untuk melihat dan menganalisis data yang dikumpulkan. Dashboard ini menyediakan visualisasi data dalam bentuk grafik, tabel, dan laporan yang mudah dipahami.

Gambar 3.6. Rancangan Game Service Frontend Dashboard

Integrasi Backend Game Service dengan Frontend Dashboard

Integrasi ini memastikan bahwa data analytics yang diproses oleh Backend Game Service dapat ditampilkan di Frontend Dashboard.

Gambar 3.7. Rancangan Integrasi Backent Game Service dengan Frontend Dashboard

Integrasi ini memastikan bahwa data analytics yang diproses oleh Backend Game Service dapat ditampilkan di Frontend Dashboard. Proses ini melibatkan pengaturan API yang memungkinkan dashboard untuk mengambil data dari server secara real-time, pengujian untuk memastikan data ditampilkan dengan benar, dan pengaturan hak akses untuk memastikan hanya pengguna yang berwenang yang dapat melihat data tertentu. Integrasi ini juga mencakup pengoptimalan performa untuk memastikan dashboard dapat menangani volume data yang besar tanpa menurunkan kinerja.

Dokumentasi Pengembangan

Dokumentasi pengembangan penting untuk memastikan bahwa sistem dapat dipelihara dan dikembangkan lebih lanjut oleh pengembang lain. Dokumentasi ini mencakup berbagai aspek, mulai dari panduan penggunaan hingga dokumentasi teknis.

Hands On Documentation

Hands On Documentation menyediakan panduan praktis bagi pengembang untuk mulai menggunakan sistem.

Contributor Documentation

Contributor Documentation memberikan panduan bagi kontributor yang ingin berkontribusi pada pengembangan sistem.

End User Documentation

End User Documentation menyediakan panduan bagi pengguna akhir yang akan menggunakan sistem untuk analitik game.

Pengujian dan Analisa Hasil

Bagian ini menjelaskan metode pengujian yang dilakukan untuk memastikan bahwa sistem berfungsi dengan baik dan memenuhi kebutuhan pengguna. Pengujian dilakukan secara menyeluruh pada backend game service dengan fitur analytics, Client SDK berbasis Unity plugin, dan Client Dashboard berbasis web. Berikut adalah teknis pengujian yang dilakukan:

Pengujian Backend Game Service

Memastikan bahwa backend game service berfungsi dengan baik, dapat menerima dan memproses data dengan benar, serta menyediakan fitur analytics yang sesuai.

Input Data Testing

Mengirim berbagai jenis data dari game client untuk memastikan data diterima dan diproses dengan benar.

API Testing

Menguji setiap endpoint API untuk memastikan respons yang dihasilkan sesuai dengan yang diharapkan.

Analytics Processing Testing

Memastikan bahwa data yang diterima dapat diproses dan dianalisis dengan benar, menghasilkan laporan dan insight yang akurat.

Load Testing

Memastikan bahwa sistem dapat menangani jumlah permintaan yang tinggi tanpa penurunan kinerja.

Pengujian Client SDK Berbasis Unity Plugin

Memastikan bahwa Unity plugin dapat berkomunikasi dengan backend game service dengan benar dan mudah digunakan oleh pengembang game.

Data Transmission Testing

Mengirim data dari Unity plugin ke backend untuk memastikan data diterima dan diproses dengan benar.

Error Handling Testing

Memastikan bahwa plugin dapat menangani error dengan baik dan memberikan pesan kesalahan yang informatif.

Pengujian Client Dashboard Berbasis Web

Memastikan bahwa Client Dashboard dapat menampilkan data analytics dengan benar dan mudah digunakan oleh pengembang game.

Data Visualization Testing

Memastikan bahwa data ditampilkan dalam bentuk grafik dan tabel yang mudah dipahami.

Filter and Sorting Testing

Memastikan bahwa fitur filter dan pengurutan data berfungsi dengan benar.

Dokumentasi Pengujian

Memastikan bahwa dokumentasi yang disediakan memadai untuk pengembang dan pengguna akhir.

Hands-On Testing

Menguji dokumentasi praktis untuk memastikan bahwa pengembang dapat mengikuti panduan dengan mudah.

Contributor Testing

Menguji dokumentasi kontributor untuk memastikan bahwa kontributor dapat memahami dan mengikuti prosedur kontribusi.

End User Testing

Menguji dokumentasi pengguna akhir untuk memastikan bahwa pengguna dapat memahami dan menggunakan layanan dengan mudah.

Analisa Hasil Pengujian

Menganalisis hasil pengujian untuk memastikan bahwa sistem berfungsi dengan baik dan memenuhi kebutuhan pengguna.

Bug Tracking

Mencatat dan menganalisis bug yang ditemukan selama pengujian untuk diperbaiki.

Performance Metrics

Mengukur dan menganalisis performa sistem berdasarkan hasil pengujian performa.

User Feedback Analysis

Mengumpulkan dan menganalisis umpan balik dari pengembang dan pengguna akhir untuk perbaikan lebih lanjut.

\clearchapter
\include{src/01-body/bab4}
\clearchapter
\include{src/01-body/bab5}
\clearchapter
%-----------------------------------------------------------------------------%
\chapter{\kesimpulan}
\label{bab:6}
%-----------------------------------------------------------------------------%

\clearchapter


% Daftar Pustaka
%
\CAPinToC
%
% Daftar Pustaka
%
% @author Atqa Munzir
% @version 1.0.0
% @edit by
%  -
%




\phantomsection %hack to add clickable section for pustaka
\bibliography{config/references}

\clearchapter
\noCAPinToC



%-----------------------------------------------------------------------------%
% Back
%-----------------------------------------------------------------------------%

% Lampiran
%
% \begin{appendix}
%   \newcounter{pagetemp}
%   \setcounter{pagetemp}{\thepage}
%   \include{_internals/markLampiran}
%   \clearchapter
%   \setcounter{page}{\thepagetemp}
%   \stepcounter{page}
%   %-----------------------------------------------------------------------------%
\addappendix{CHANGELOG}
\begin{flushright}
  Lampiran 1: CHANGELOG
\end{flushright}
\label{appendix:changelog}
%-----------------------------------------------------------------------------%

% \end{appendix}


% Biodata Penulis
%
\addChapter{Biodata}
%
% Lembar Biodata Penulis
%
% @author Atqa Munzir
% @version 1.0.0
% @edit by
%  -
%


%-----------------------------------------------------------------------------%
\chapter*{BIODATA}
%-----------------------------------------------------------------------------%





\end{document}
