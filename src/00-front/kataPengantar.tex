%
% Lembar Kata Pengantar
%
% @author Atqa Munzir
% @version 1.0.0
% @edit by
%  -
%


%-----------------------------------------------------------------------------%
\chapter*{\KataPengantar}
%-----------------------------------------------------------------------------%


{
  \hbadness=10000

  \hspace*{1.0cm}
  Puji syukur kehadirat Tuhan Yang Maha Esa atas segala rahmat dan karunia-Nya sehingga penulis dapat menyelesaikan laporan \type\ ini dengan baik. Laporan ini disusun sebagai salah satu syarat untuk memenuhi tugas akhir pada program studi \jenjang\ \prodi\ di Politeknik Elektronika Negeri Surabaya.

  Penulis menyadari bahwa dalam penyusunan laporan ini tidak lepas dari bantuan dan dukungan berbagai pihak. Oleh karena itu, pada kesempatan ini, penulis ingin menyampaikan ucapan terima kasih yang sebesar-besarnya kepada:

  \begin{enumerate}[topsep=0pt,itemsep=-1ex,partopsep=0ex,parsep=1ex]
    \item Allah SWT atas limpahan rahmat dan ridha-Nya sehingga penulis dapat menyelesaikan tugas akhir ini.
    \item Rasulullah Muhammad SAW sebagai suri tauladan penulis dan sebagai sumber inspirasi bagi penulis.
    \item Orang tua dan keluarga yang selalu memberikan dukungan, doa, dan bantuan selama pengerjaan tugas akhir ini.
    \item Bapak \kajur, selaku Kepala Jurusan \jurusan.
    \item Bapak \kaprodi, selaku Ketua Program Studi \prodi\ beserta dosen pembimbing dua selama tugas akhir.
    \item Bapak \pembimbingSatu, selaku dosen pembimbing satu selama tugas akhir.
    \item Seluruh pihak lain yang telah membantu kesuksesan tugas akhir ini, meskipun tidak dapat disebutkan satu persatu.
  \end{enumerate}

  Penulis menyadari bahwa laporan \type\ ini masih jauh dari sempurna. Oleh karena itu, apabila terdapat kesalahan atau kekurangan dalam laporan ini, Penulis memohon agar kritik dan saran bisa disampaikan langsung melalui \f{e-mail} \code{atqamz@gmail.com}.

  Akhir kata, semoga laporan tugas akhir ini dapat memberikan manfaat bagi semua pihak yang membacanya dan dapat menjadi referensi yang bermanfaat bagi peneliti-peneliti berikutnya.
}

% Untuk input gambar tanda tangan, silahkan sesuaikan xshift, yshift, dan width dengan gambar tanda tangan Anda
%\begin{tikzpicture}[remember picture,overlay,shift={(current page.north east)}]
%\node[anchor=north east,xshift=-3cm,yshift=-6.2cm]{\includegraphics[width=3cm]{assets/pics/tanda_tangan_wikipedia.png}};
%\end{tikzpicture}

\begin{flushright}
  Surabaya, \tanggalSiapSidang \\ [0.1cm]
  \vspace*{1.0cm}
  \penulis
\end{flushright}

\newpage
